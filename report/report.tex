%%%%%%%% ICML 2026 EXAMPLE LATEX SUBMISSION FILE %%%%%%%%%%%%%%%%%

\documentclass{article}

% Recommended, but optional, packages for figures and better typesetting:
\usepackage{microtype}
\usepackage{graphicx}
\usepackage{subcaption}
\usepackage{booktabs}
\usepackage{hyperref}
\newcommand{\theHalgorithm}{\arabic{algorithm}}

\usepackage[preprint]{icml2026}

\usepackage{amsmath}
\usepackage{amssymb}
\usepackage{mathtools}
\usepackage{amsthm}
\usepackage[capitalize,noabbrev]{cleveref}

%%%%%%%%%%%%%%%%%%%%%%%%%%%%%%%%
% THEOREMS
%%%%%%%%%%%%%%%%%%%%%%%%%%%%%%%%
\theoremstyle{plain}
\newtheorem{theorem}{Theorem}[section]
\newtheorem{proposition}[theorem]{Proposition}
\newtheorem{lemma}[theorem]{Lemma}
\newtheorem{corollary}[theorem]{Corollary}
\theoremstyle{definition}
\newtheorem{definition}[theorem]{Definition}
\newtheorem{assumption}[theorem]{Assumption}
\theoremstyle{remark}
\newtheorem{remark}[theorem]{Remark}

\usepackage[textsize=tiny]{todonotes}

\icmltitlerunning{Statistical Learning Based Pediatric Appendicitis Prediction Project}

\begin{document}

\twocolumn[
  \icmltitle{Statistical Learning Based Pediatric Appendicitis Prediction Project}
  
  \icmlsetsymbol{equal}{*}

  \begin{icmlauthorlist}
    \icmlauthor{Zitong Wang}{equal,pkusms}
    \icmlauthor{Wen Yuan}{equal,pkusms}
    \icmlauthor{Jingxing Zhou}{equal,pkusms}
  \end{icmlauthorlist}

  \icmlaffiliation{pkusms}{School of Mathematical Sciences, Peking University}

  \icmlcorrespondingauthor{Zitong Wang}{2300010750@stu.pku.edu.cn}
  \icmlcorrespondingauthor{Wen Yuan}{2200010833@stu.pku.edu.cn}
  \icmlcorrespondingauthor{Jingxing Zhou}{2300010749@stu.pku.edu.cn}

  \icmlkeywords{Statistical Learning, Pediatric Appendicitis}

  \vskip 0.3in
]

\printAffiliationsAndNotice{\icmlEqualContribution}

\begin{abstract}
  This document provides a basic paper template and submission guidelines.
  Abstracts must be a single paragraph, ideally between 4--6 sentences long.
  Gross violations will trigger corrections at the camera-ready phase.
\end{abstract}

\section{Introduction}

TBD

\section*{Accessibility}

TBD

\section*{Software and Data}

TBD

\section*{Acknowledgements}

TBD

\section*{Impact Statement}

This paper presents work whose goal is to advance the field of Machine
Learning. There are many potential societal consequences of our work, none
which we feel must be specifically highlighted here.

% In the unusual situation where you want a paper to appear in the
% references without citing it in the main text, use \nocite
\nocite{langley00}

\bibliography{report}
\bibliographystyle{icml2026}

%%%%%%%%%%%%%%%%%%%%%%%%%%%%%%%%%%%%%%%%%%%%%%%%%%%%%%%%%%%%%%%%%%%%%%%%%%%%%%%
%%%%%%%%%%%%%%%%%%%%%%%%%%%%%%%%%%%%%%%%%%%%%%%%%%%%%%%%%%%%%%%%%%%%%%%%%%%%%%%
% APPENDIX
%%%%%%%%%%%%%%%%%%%%%%%%%%%%%%%%%%%%%%%%%%%%%%%%%%%%%%%%%%%%%%%%%%%%%%%%%%%%%%%
%%%%%%%%%%%%%%%%%%%%%%%%%%%%%%%%%%%%%%%%%%%%%%%%%%%%%%%%%%%%%%%%%%%%%%%%%%%%%%%
\newpage
\appendix
\onecolumn
\section{You \emph{can} have an appendix here.}

You can have as much text here as you want. The main body must be at most $8$
pages long. For the final version, one more page can be added. If you want, you
can use an appendix like this one.

The $\mathtt{\backslash onecolumn}$ command above can be kept in place if you
prefer a one-column appendix, or can be removed if you prefer a two-column
appendix.  Apart from this possible change, the style (font size, spacing,
margins, page numbering, etc.) should be kept the same as the main body.
%%%%%%%%%%%%%%%%%%%%%%%%%%%%%%%%%%%%%%%%%%%%%%%%%%%%%%%%%%%%%%%%%%%%%%%%%%%%%%%
%%%%%%%%%%%%%%%%%%%%%%%%%%%%%%%%%%%%%%%%%%%%%%%%%%%%%%%%%%%%%%%%%%%%%%%%%%%%%%%

\end{document}
